\documentclass[
  a4paper,
  12pt,
  parskip=half,
  headings=standardclasses,
  footskip=0pt,
  footlines=1,
  headheight=80in
]{scrartcl}
\setlength{\textheight}{1.15\textheight}

\input{metainfo-include.tex}

\sloppy
\usepackage[utf8]{inputenc}
\usepackage{enumerate}
\usepackage{xcolor}
\usepackage{graphicx}
\usepackage{hyperref}
\usepackage{amsmath}
\usepackage{amssymb}
\usepackage{paracol}
\usepackage{palatino}
\usepackage{mathtools}
\usepackage{listings}
\usepackage[headsepline]{scrlayer-scrpage}
\usepackage[most, listings]{tcolorbox}
\usepackage{longtable}[=v4.13]
\usepackage{tabu}
\usepackage{tikz}
\usetikzlibrary{shapes.misc, positioning}

\definecolor{darkblue}{rgb}{0,0,0.5}
\hypersetup{%
  colorlinks=true,
  urlcolor=darkblue
}

\pagestyle{scrheadings}
\setkomafont{pageheadfoot}{}
\ohead{\small CP 2023}
\ihead{\small Prof.\ Dr.\ Friedrich, Dr.\ Lenzner, Fischbeck, Gawendowicz}
\cfoot{}

\addtolength{\headheight}{-1.5em}

\newcommand{\makeheader}{%
  \columnratio{0.2,0.6,0.2}
  \begin{paracol}{3}
      \switchcolumn
  \center{\Large\textbf{Competitive Programming SS23}}\\
  \vspace{2mm}
  \textbf{Submit until end of contest}
  \switchcolumn
  \hfill
  \includegraphics[width=2.5cm]{hpi-logo.pdf}\\
  \end{paracol}

  \noindent\textbf{Problem: \problemName{}} (\timelimit{} second timelimit) 
}
% \newcommand{\makeheader}{%
%   \columnratio{0.7}
%   \begin{paracol}{2}
%     \begin{leftcolumn}
%       \noindent{\Large\textbf{(Adv.) Competitive Programming}}\\

%       \noindent{\small\textbf{Submit until \deadline{}}}
%     \end{leftcolumn}
%     \begin{rightcolumn}
%       \vspace*{-8mm}
%       \includegraphics[width=4cm]{Hasso_Plattner_Institut_Logo.pdf}
%     \end{rightcolumn}
%   \end{paracol}
%   \vspace*{1em}

%   \noindent\textbf{Problem: \problemName{}} (\timelimit{} second timelimit)
% }

\newcommand{\placeholder}[1]{\textcolor{blue}{#1}}

\newenvironment{samples}[1][1]{%
  \noindent
  \begin{longtabu} to \textwidth {@{}X[#1] X[1]@{}}
    \textbf{Sample input} & \textbf{Sample output} \\
}{%
  \end{longtabu}
}
\newcommand{\sampleBox}[1]{%
  % Using frame=empty doesn't work for listings broken across pages
  \tcbinputlisting{listing file=#1, listing only, colframe=white, colback=black!10!white, sharp corners, box align=top}
}
\newcommand{\sample}[1]{%
  \sampleBox{../testcases/#1.in} & \sampleBox{../testcases/#1.ans} \\
}

\newcommand{\bonusnotice}{%
\textit{Note:} This is a problem that is harder to solve than usual.
Solve the other problems first before spending too much time on this one.

\vspace{1em}
}


\begin{document}

\makeheader

\placeholder{Problem description} 
\\
In 2023 Hasso Plattner is launching his own currency (HPC). In the basement of the newly build buildings at Campus 2 the money press will be guarded by a complex alarm system. 
You want to be able to print your own HPC and therefore you need the money printing plates from the basement. 
In the basement is a long hallway of dimensions (width 10m and length 100m) separates you from money press. There are n motion detectors on the ceiling that detect intruders if they pass within a circular range of r. 
Your goal is to find out, if it is possible to reach the other end of the hallway without triggering the alarm. 

\paragraph*{Input}
\placeholder{Input description}
n: an integer that describes the number of motion detectors \\
followed by n lines of:
$x_i$, $y_i$ and $r_i$: the position and the range of the ith motion detector


\paragraph*{Output}
\placeholder{Output description}
“Yes” if it is possible to reach the other end of the hallway \\
“No” if it is not possible to reach the other end of the hallway


\begin{samples}
  \sample{sample1}
  \sample{sample2}
  \sample{sample3}
  \sample{sample4}
\end{samples}

\placeholder{Solution sketch}
\\
Every motion detector is a node in a graph, furthermore there is a node $n_upper$ and $n_lower$. 
An alarm node $n_i$ is connected to another alarm node $n_j$ if they are at most $r_i$ + $r_j$ apart from each other. 
$n_upper$ is connected to each alarm node $n_i$ that is at most $r_i$ apart from upper border, the same hold for $n_lower$ with respect to the lower border. 
On that graph we execute a dfs or bfs with $n_upper$ as the start node. If $n_lower$ is reachable, there is a line of motion detectors that spans from top to bottom, thus making it impossible to reach the other side of the room.

\end{document}